\documentclass{article}
\usepackage{url}
\newcommand{\ccnsim}{ccn$\mathcal{S}$im}

\begin{document}
\section{Introduction}
\subsection{What is \ccnsim?}
\ccnsim\ is a scalable chunk-level simulator of Content Centric Networks (CCN)\cite{jacobson09conext}, that we developed in the context of the ANR Project Connect.  

\begin{itemize}
    \item It is written in C++ under the Omnet++ framework.
    \item It  allows to simulate CCN networks in scenarios with large orders of magnitude.
    \item It is distributed as free software, downloadable at \url{http://site}.
\end{itemize}
\ccnsim\ extends Omnet++ as to provide a modular environment in order to simulate CCN networks. Mainly,  \ccnsim\ models the forwarding aspects of a CCN network, namely the caching strategies, and the forwarding layer of a CCN node. However, it is fairly modular, and simple to . We hope that you enjoy \ccnsim\, in which case we ask you to please cite our paper \cite{ccn12icc}. 
\ccnsim\ is able to simulate content stores up to 106 chunks and catalog sizes up to 108 files in a reasonable time.  
%for CCN content stores and Internet catalog sizes On a PC equipped with 24GB of RAM, 
\subsection{Organization of this manual}
This manual is organized as follows:
\begin{itemize}
    \item Organization.
\end{itemize}

\section{Compiling \ccnsim}
We assume that you have downloaded and installed Omnetpp (version $\geq$ 4.1) in
your home directory. In order to run \ccnsim\, it is first necessary to patch
Omnetpp (in order to use the multi-path extension described in [1]), as follows:
%• Note 1: the version 0.1 of \ccnsim\ requires a 64bit kernel. We have suc-
%cessfully built and run \ccnsim\-0.1 with Ubuntu versions as old as 8.04
%(provided that the parallelism is architecture 64bits) and use a Linux
%2.6.32 kernel on a Ubuntu 10.04LTS for our simulations. We have not
%tested 32bits architectures, nor will support them (see Note 2).
%• Note 2: future versions of \ccnsim\ (currently under testing) will remove
%the 64bits limitation by making use of the Boost libraries.
%3
%Hello, (CCN) world!
%Once you compiled the simulator, you’re able to run your first CCN simulation.
%The syntax to invoke the simulator execution is the following:
%\ccnsim\@sushi:~/\ccnsim\-0.1$ ./\ccnsim\-0.1
%%-u Cmdenv
%%youtube.ini
%%To start, we provide a youtube.ini settings file, that specify a VoD catalog as
%%in [1, 2, 4]. out0.node_local
%%• Note 1: In the above syntax, you can omit the configuration file name in
%%case you store your settings in omnetpp.ini rather than youtube.ini.
%%• Note 2: The youtube.ini file contains comments about the variables
%%involved, to which we refer the reader for more detailed comments.
%%• Note 3: For indication, running a 1800 seconds of simulated time of
%%YouTube catalog on a Geant network (i.e., the default scenario in youtube.ini)
%%takes about 2400 seconds of real time on an Intel Xeon E5620 @ 2.40GHz
%%equipped with 12MB cache and 24GB ram.
%%4
%%Simulation input
%%In omnetpp the scenario is described through “ini” files. We report here the
%%example youtube.ini shipped along with the simulator sources, containing de-
%%tailed and hopefully useful comments to modify the scenario description.
%%#===============================================
%%#YouTube-like catalog configuration for \ccnsim\
%%#-----------------------------------------------
%%# Topologies of CCN network.
%%[General]
%%network = topologies.geant_network
%%#===============================================
%%# output
%%#-----------------------------------------------
%%2
%%# By default, output in results/
%%output-vector-file = ${resultdir}/out${repetition}.vec
%%output-scalar-file = ${resultdir}/out${repetition}.sca
%%# number of runs
%%repeat = 1
%%seed-set = ${repetition}
%%#===============================================
%%# simulation duration
%%#-----------------------------------------------
%%# There are three simulation states:
%%# 1) filling caches
%%# 2) transient
%%# 3) steady-state
%%#
%%# 1) you can start with empty caches (warmup=false,default) or
%%# with caches already filled with randomized content (warmup=true).
%%# Starting with filled caches is useful only when Zipf exponent
%%# s very large (say >1.5) so that caches may not even fill up
%%# due to very skewed requests.
%%#
%%**.warmup = false
%%# 2) parameters tuning intervals of confidence estimation for hit
%%# rate of individual caches, affecting convergence speed;
%%#
%%# We implement two convergence methods (avg|wc):
%%# -Average (avg): we require the convergence of the network hit rate
%%# -Worst Case(wc): we require the convergence of all individual nodes
%%#
%%**.convergence_type = "wc"
%%# Zipf percentile limit for stabilization node; i.e., we require the
%%# convergence statistics only for the bulk of the catalog (90% by default)
%%**.convergence_zipf_percentile = 0.9
%%# threshold of variance hit rate (wc=per node, avg=over all nodes)
%%**.convergence_threshold_wc = 0.001
%%**.convergence_threshold_avg = 0.1
%%# 3) simulated time to spend in steady state after transient ended
%%**.sim_time = 1800
%%3
%%#===============================================
%%# Catalog
%%#-----------------------------------------------
%%# Global request rate, in Hz (over all network nodes).
%%**.lambda = 50
%%# Client requests can be geographically skweked (see [2] in HOWTO).
%%# When D==-1 uniform requests are considered (default).
%%**.D = -1
%%# Content popularity: Mandelbrot-Zipf distribution parameters
%%# Mandelbrot-Zipf exponent
%%**.alpha = 1
%%# Mandelbrot-Zipf plateau
%%**.q = 0
%%# Catalog size, in files
%%**.F = 10^8
%%# Average file size in chunks, geometrically distributed
%%**.avgC = 1000
%%# Number of replicas of original (ie, non-cached) content.
%%**.repository_num = 1
%%# Repository placement policy
%%# There are two policies (ugc,pop)
%%# -ugc = user generated content,
%%# Multiple replicas may exist, but there is no single repository
%%# storing the whole catalog. Rather, individual files are randomly
%%# assigned to a repository, that can be located behind any
%%# CCN router (i.e., in practice we have as many repositories
%%# as CCN nodes).
%%# -pop = point of presence,
%%# The number of repositories equal the number of replicas, and
%%# the original replicas of the whole catalog are stored in each
%%# repository.
%%#
%%#
%%**.repository_policy = "pop"
%%#===============================================
%%# CCN node
%%#-----------------------------------------------
%%4
%%# Cache size (in chunks)
%%**.S = 10^6
%%# Caching Replacement Policy {(lru)|rnd|fifo|two}, see [1] in HOWTO
%%**.cache_replacement_policy = "lru"
%%# Decision Policy {(always)|never|distance|lcd|fixP}, see [1] in HOWTO
%%**.decision_policy = "always"
%%# Strategy layer policy {closest|all}, see [1] in HOWTO;
%%# other interest forwarding policies will be available in
%%# future releases of \ccnsim\
%%**.strategy_layer = "closest"
%%# Multipath cardinality,
%%# M=1 => shortest path routing
%%# M=2 => two disjoint paths, valid only for strategy_layer=closest, see [1] in HOWTO
%%**.M = 1
%%# Multipath performance,
%%# Flag determining if paths are precomputed and stored in an external file
%%# (under the directory "optimal") or if they have to be calculated on demand.
%%# Path stored in file are faster, but they need to be computed once for any
%%# new topology.
%%**.fromFile = true
%%# Discouraged parameter, will be removed from future versions of \ccnsim\.
%%# (do not modify but do not delete either).
%%**.c = 0
%%5
%%Simulation output
%%Once you have run the simulation as in the previous paragraph, you will find
%%the following statistics. Output is organized in a number of files, following
%%an outID.extension naming notation where ID is the repetition number and
%%extension can be among the following:
%%In file out0.sca:
%%• DATA[i], INT[i]: number of data/interest messages handled by node i
%%during the simulation;
%%• hl[i]: chunk-level hit rate at different distances with respect to the request
%%originator (i.e., tied with the distance at which the chunk has been found);
%%5
%%• stretch: normalized distance with respect to the server storing persistent
%%replicas, see [1, 2];
%%• cache hit: average cache hit rate over all CCN nodes (see out0.node_local
%%for individual hit rates);
%%• full time: time to fill all caches (i.e., after which the transient starts);
%%• transient time: time at which the transient ends (and the steady state
%%starts, that lasts for the amount of time specified in youtube.ini);
%%• div ratio: cache diversity ratio (see [1]).
%%In file out0.node_local:
%%• List of repositories with cardinality (i.e., number of replicas)
%%• For each CCN node i the correspondent final HitRate achieved at the end
%%of the simulation.
%%In file out0.node_stat:
%%• For each file k downloaded from node i, the log contains three values,
%%namely k, the effort η (see definition in [2]) i.
%%In file out0.vec:
%%• If the convergence criterion of the simulation is the average global hit rate
%%(see youtube.ini), this file reports the time evolution of the average hit
%%rate (0.1 second steps) until the stabilization.
%%6
%%Extending \ccnsim\
%%As our main effort is devoted to extending \ccnsim\, please consider that we have
%%limited effort to spend on its support (e.g., manual, code comments, etc.),
%%Hence, you should consider \ccnsim\ is a “PhD-grade” simulator rather than
%%a “turnkey” simulator. This means that (at least) MSc- (or better) PhD-grade
%%skills may be necessary to run and modify the simulator, and inspection of the
%%code may help over the lack of documentation (blame on us).
%%For any questions, comments, suggestions, please use the \ccnsim\ mailing list
%%ccnsim@lincs.fr rather than our personal addresses.
\bibliographystyle{plain}
\bibliography{manual.bib}
\end{document}
