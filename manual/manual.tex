\documentclass{article}
\usepackage{url}
\usepackage{fancyvrb}
\newcommand{\ccnsim}{ccn$\mathcal{S}$im}

\begin{document}
\tableofcontents
\newpage
\section{Introduction}
\subsection{What is \ccnsim?}
\ccnsim\ is a scalable chunk-level simulator of Content Centric Networks (CCN)\cite{jacobson09conext}, that we developed in the context of the ANR Project Connect.  

\begin{itemize}
    \item It is written in C++ under the Omnet++ framework.
    \item It  allows to simulate CCN networks in scenarios with large orders of magnitude.
    \item It is distributed as free software, downloadable at \url{http://site}.
\end{itemize}
\ccnsim\ extends Omnet++ as to provide a modular environment in order to simulate CCN networks. Mainly,  \ccnsim\ models the forwarding aspects of a CCN network, namely the caching strategies, and the forwarding layer of a CCN node. However, it is fairly modular, and simple to . We hope that you enjoy \ccnsim\, in which case we ask you to please cite our paper \cite{ccn12icc}. 
\ccnsim\ is able to simulate content stores up to 106 chunks and catalog sizes up to 108 files in a reasonable time.  
%for CCN content stores and Internet catalog sizes On a PC equipped with 24GB of RAM, 
\subsection{Organization of this manual}
This manual is organized as follows:
\begin{itemize}
    \item Organization.
\end{itemize}


\section{Downloading and installing \ccnsim}
You can freely download \ccnsim\ from the project site: \url{http://code.google.com/p/ccnsim/}. 

Moreover, we assume that you have downloaded and installed Omnetpp (version $\geq$ 4.1) on your machine. In order to install \ccnsim, it is first necessary to patch Omnetpp. Then you can compile the \ccnsim\ sources. These steps are as follows:

\begin{Verbatim}[frame=single]
    john:~$ cd CCNSIM_DIR
    john:CCNSIM_DIR$ cp ./patch/ctopology.h OMNET_DIR/include/
    john:CCNSIM_DIR$ cp ./patch/ctopology.cc OMNET_DIR/src/sim
    john:CCNSIM_DIR$ cd  OMNET_DIR && make && cd CCNSIM
    john:CCNSIM_DIR$ ./scripts/makemake
    john:CCNSIM_DIR$ make
\end{Verbatim}
We suppose that \verb|CCNSIM_DIR| and \verb|OMNET_DIR| contain the installation directory of \ccnsim\ and Omnet++ respectively. 
\section{Overview}
\subsection{Overall structure of \ccnsim}
In order to better understand the organization of \ccnsim, the best is to look at its internal organization. In the following we reproduce the basic directory organization of \ccnsim. 
\begin{Verbatim}[frame=single]
|-- topologies
|-- modules
|   |-- clients
|   |-- content
|   |-- node
|   |   |-- cache
|   |   |-- strategy
|   |-- statistics
|-- packets
|-- include
|-- src
|   |-- clients
|   |-- content
|   |-- node
|   |   |-- cache
|   |   |-- strategy
|   |-- statistics
\end{Verbatim}
As said within the introduction, \ccnsim\ is a package built over the top of Omnet++. As such, we can divide its implementation in two different subunits. One subunit is represented by the \verb|.ned| description of the modules used by \ccnsim, and included within the directory \verb|modules| and \verb|topologies|. The first directory, is basically the \verb|.ned| description of the modules employed by \ccnsim, like clients, nodes, and so forth. Within the \verb|topologies| directory there are some sample topologies (always descripted in \verb|.ned| format) ready to be used. 

The real implementation of these modules lie within the \verb|src| and \verb|include| directory, that contain sources and header files, respectively.  Within the rest of this section we summarily  describe the features of these components, together with a brief overview of their most important parameters.

\subsection{Nodes, caches, and strategy layers}
Nodes are the core part of the \ccnsim. They form the core network and can be connected each other by the means of \verb|faces|(see \cite{jacobson09conext} for better understanding what a \emph{face} is). However, nodes own not a real C++ implementation. A node is just a compound module, composed by three parts: the Core Layer, the Strategy Layer, and the Content Store.
\subsubsection{Core Layer} 
The core layer implements the basic tasks of a CCN node. Indeed, it handles the PIT, sending data back to the interested interfaces. In handles the incoming interest by replying to it (in the case of a \emph{cache hit}), or by  appending the interest to the existent PIT entry. In the case no entry exists yet it dispatches it to the \emph{Strategy layer} in order to get a decision about where (i.e., on which interface) sending it. 
\subsubsection{Strategy Layer}
The strategy layer receives an interest for which no PIT entry exists yet. Then, it has to \emph{decide} where sending the given interest over the network. 
\subsubsection{Content Store}

\subsection{The clients}
\subsection{The content distribution}
\subsection{Statistics}
\subsection{Global parameters}

\section{Run your first simulation}
\section{Extending \ccnsim}
\bibliographystyle{plain}
\bibliography{manual.bib}
\end{document}
